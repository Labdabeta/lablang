% TODO: Consider no memory mapped I/O, but port mapped I/O instead.

\documentclass{article}

\usepackage[margin=0.5in]{geometry}

\usepackage{amsmath}
\usepackage{array}
\usepackage{booktabs}
\usepackage{caption}
\usepackage{courier}
\usepackage{float}
\usepackage{hyperref}
\usepackage{listings}
\usepackage[usenames]{xcolor}

\lstdefinelanguage{lab}
{keywords={add,mul,div,mod,and,nan,orr,xor,bgt,beq,cas,bal,ldr,str,get,put,clz,pop,shr,mov,itf,flr,cel,rnd,fad,fmu,fdv,fmo,fbg,fbe,ldf,stf},
 otherkeywords={r0,r1,r2,r3,r4,r5,r6,r7,fp,sp,(,)},
 sensitive=false,
 comment=[l]{;},
}

\definecolor{grey}{rgb}{0,0.6,0}
\lstset{
    basicstyle=\ttfamily,
    keywordstyle=\color{blue},
    commentstyle=\color{grey}
}

\newcommand{\labcode}[1]{\colorbox{lightgray}{\lstinline[language=lab]{#1}}}

\title{LAB Processor -- Documentation}
\date{\today}
\author{Louis A. Burke}

\begin{document}
\maketitle\clearpage

\section{Introduction}

The LAB processor is a general-purpose reduced instruction set processor design.
Its intended purpose is to provide a simple library to emulate a processor for
educational purposes. It is packaged with two implementations, one is a 32-bit
system with $2^21$ words of addressable memory. The other is a 64-bit system
with $2^53$ words of addressable memory. Both systems support multiple
processors, interrupt tables, and memory mapped addressing. Additionally the
implementations can simulate different branch predictors, caches, and pipelines.
Effectively every aspect of the processor can be fine-tuned when using the
provided implementations, however sensible defaults are provided.

In addition to the provided emulation implementations, there is also an
assembler, C compiler, and linker which can be used to build binaries for LAB.

\section{Registers}

The LAB processor has 8 registers, three of which are special-purpose. They are
summarized in table \ref{table:registers}.

\begin{table}[h!]
\centering
\begin{tabular}{ll}
    \toprule Register & Description \\ \midrule
    \labcode{r0} & Read-only, returns 0 \\
    \labcode{r1} .. \labcode{r5} & General purpose \\
    \labcode{fp} (\labcode{r6}) & Frame pointer (works as general purpose) \\
    \labcode{sp} (\labcode{r7}) & Stack pointer (initialized to highest address)
\end{tabular}
\caption{LAB Registers}
\label{table:registers}
\end{table}

In addition to these 8 registers, there is also the hidden program counter
register. This is only modifiable via branch instructions.

\section{Operations}

The LAB processor has 32 operations. Every operation takes two registers and an
immediate value. The only difference between the 64 and 32 bit systems is how
large the immediate value can be. Their 3-letter mnemonics and descriptions
are listed in table \ref{table:operations}. The registers are named $A$ and $B$
while the immediate value is named $I$. Memory access is written as
$[x]$.

\begin{table}[h!]
\centering
\begin{tabular}{cclll}
    \toprule Group & Op-code & OP & Description \\ \midrule
    Arithmetic  & 00000 & \labcode{ADD} & $A := A + B + I$ \\
                & 00001 & \labcode{MUL} & $A := A(B + I)$ \\
                & 00010 & \labcode{DIV} & $A := \frac{A}{B + I}$ \\
                & 00011 & \labcode{MOD} & $A := A \mod (B + I)$ \\ \midrule
    Bitwise     & 00100 & \labcode{AND} & $A := A \land (B + I)$ \\
                & 00101 & \labcode{NAN} & $A := A \uparrow (B + I)$ \\
                & 00110 & \labcode{ORR} & $A := A \lor (B + I)$ \\
                & 00111 & \labcode{XOR} & $A := A \oplus (B + I)$ \\ \midrule
    Branches    & 01000 & \labcode{BGT} & if $A > B$ goto $I$ \\
                & 01001 & \labcode{BEQ} & if $A = B$ goto $I$ \\
                & 01010 & \labcode{CAS} & if $[I] = A$ then $A := 1; [I] = B$ else $A := 0$ \\
                & 01011 & \labcode{BAL} & $A := PC$ goto $B + I$ \\ \midrule
    Memory      & 01100 & \labcode{LDR} & $A := [B + I]$ \\
                & 01101 & \labcode{STR} & $[B + I] := A$ \\
                & 01110 & \labcode{GET} & $A := read(I + B)$ \\
                & 01111 & \labcode{PUT} & $write(I + B) := A$ \\ \midrule
    Bit-Manip   & 10000 & \labcode{CLZ} & $A := \mbox{clz}(B + I)$ \\
                & 10001 & \labcode{POP} & $A := \mbox{popcnt}(B + I)$ \\
                & 10010 & \labcode{SHR} & $A := \frac{A}{2^{B + I}}$ \\
                & 10011 & \labcode{MOV} & $A := B + I$ \\ \midrule
    Float I     & 10100 & \labcode{ITF} & $A := \mbox{float}(B + I)$ \\
    (Conversion)& 10101 & \labcode{FLR} & $A := \lfloor B + I \rfloor$ \\
                & 10110 & \labcode{CEL} & $A := \lceil B + I \rceil$ \\
                & 10111 & \labcode{RND} & $A := \lfloor B + I + 0.5 \rfloor$ \\ \midrule
    Float II    & 11000 & \labcode{FAD} & $A := A + B + I$ \\
    (Arithmetic)& 11001 & \labcode{FMU} & $A := A(B + I)$ \\
                & 11010 & \labcode{FDV} & $A := \frac{A}{B + I}$ \\
                & 11011 & \labcode{FMO} & $A := A - (B + I) \lfloor \frac{A}{B + I} \rfloor$ \\ \midrule
    Float III   & 11100 & \labcode{FBG} & if $A > B$ goto $I$ \\
    (Control)   & 11101 & \labcode{FBE} & if $A = B$ goto $I$ \\
                & 11110 & \labcode{LDF} & $A := [B + I]$ \\
                & 11111 & \labcode{STF} & $[B + I] := A$ \\ \midrule
\end{tabular}
\caption{LAB Operations}
\label{table:operations}
\end{table}

\section{Encoding}

The most significant 5 bits of the operation oncode the operation in use. The
following three bits encode the $A$ register, the three after that the $B$
register. The remaining bits encode the $I$ value. The immediate can be of
different lengths depending on the size of the operations (32 or 64-bit). The
immediate value is always unsigned. For reference, here is a bit-packing
representation with most significant on the left, and least significant on the
right:

% TODO: maybe a status bit and more ops (vector stuff?)
% TODO: maybe smaller instruction encoding size (13-bit max immediate??)
% TODO: PROBABLY - a simple 'core' that can be hooked up (2nd line, 1st given
% via get/put) perhaps mimic ARM more (EG: more special registers? PC accessible
% can be useful?)
% Encoding:  BB: 00 = whole, 01 = low half, 10 = high half, 11 = byte, S: signed
% SBBNNNNN AAAABBBB IIIIIIII IIIIIIII
\texttt{NNNNNAAA BBBIIIII IIIIIIII IIIIIIII \textcolor{lightgray}{IIIIIIII IIIIIIII IIIIIIII IIIIIIII}}

\section{Pipeline}

% TODO

\section{I/O}

% TODO

\section{Interrupts}

The LAB processor uses a Vectored Interrupt Controller (VIC). It supports two
kinds of interrupts. The first are the \textit{Interrupt Requests} (IRQs). The
second are \textit{Fast Interrupt Requests} (FIQs). All FIQs have a higher
priority than all IRQs. Additionally while servicing an FIQ no other interrupt
may occur, meanwhile during processing of an IRQ an interrupt of lower priority
could be called.

The priorities are all from least significant bit to highest significant bit
within the status registers (see below).

The VIC utilizes a few key memory-mapped registers. These are relative to the
VIC base address. This address is the first not accessible via the immediate
value. Thus on 32-bit machines it is \texttt{0x20\_0000} while on 64-bit
machines it is \texttt{0x20\_0000\_0000\_0000}.

% TODO: HERE: Describe how the bits in these registers are packed
% TODO: make this into a table with clicky links to descriptions

The first VIC register is at offset 0 (exactly at the VIC base address). It is
the IRQ status register. Its value contains a one at each bit position that
is enabled, selected for IRQ, and active. Each of these ones will generate an
interrupt until all are cleared. The IRQ status register is read-only.

The second VIC register is at offset 1. It is the FIQ status register. Its
value contains a one at each bit position that is enabled, selected for FIQ, and
active. Each of these ones will generate an interrupt until all are cleared. The
FIQ status register is read-only.

The third VIC register is at offset 2. It is the raw interrupt status register.
Its value contains a one at each bit position that is active. The raw interrupt
status register is read-only.

The fourth VIC register is at offset 3. It is the interrupt select register. Its
one bits represent interrupts that will be handled by FIQs while the zero bits
represent interrupts that will be handled by IRQs.

The fifth VIC register is at offset 4. It is the interrupt enable register. Its
one bits represent interrupts that can be handled by the interrupt handler.

The sixth VIC register is at offset 5. It is the interrupt enable clear
register. Any bits written to one on it will be set to zero on the interrupt
enable register. The interrupt enable clear register is write-only.

The seventh VIC register is at offset 6. it is the software interrupt register.
Any bit written to 1 on it will generate the corresponding interrupt.

The eighth VIC register is at offset 7. It is the software interrupt clear
register. Any bit written to 1 on it will set the corresponding bit to 0 on the
software interrupt register.

The ninth VIC register is at offset 8. It is the protection register. If the
least significant bit is set to 1, then all VIC registers can only be accessed
in protected mode. It is only accessible from protected mode.

The tenth VIC register is at offset 9. It is the vector address register. It
contains the address of the currently operating interrupt service routine
(ISR). If the value in this register is read it signals that processing of the
interrupt is starting. No lower priority interrupts can occur until any value is
written to this address again.

The eleventh VIC register is at offset 10. It is the default vector address
register. It contains the address of the default ISR.

The twelfth to fourty-fourth VIC registers are at offsets 11 to 43. They
correspond to the address of the ISR for the associated interrupt. If set to 0,
they will not be called and the default will be used instead.

Which interrupts correspond to which external events is implementation defined.
It is often linked to external devices which are accessed with the \labcode{GET}
and \labcode{PUT} instructions. The only exception is interrupt 0, which is
caused by a protection violation.

\section{Protection}

The LAB processor has two protection modes. It starts in protected mode and
switches to user mode when the protection status register is written. The
protection status register is located at memory address 1.

In user mode writing to the protection status register will jump to the address
in the protection address register (memory address 0) and switch to protected
mode. All values in registers persist across a mode change.

In protected mode whatever value is written to the protection status register
will be the minimum usable memory address for the user. Any access to memory
lower than the value stored will trigger a protection violation (or accessing
the VIC registers if they are protected).

\section{LAB Assembly}

The provided assembler accepts commands in the \labcode{OP A, B (I)} format. The
immediate value is optional, but the registers are not. The immediate value can
be any basic literal expression. The basic literal expressions can consist of
addition, subtraction, multiplication, division, and parentheses. Additionally
it may refer to labels in the code or the special ``current instruction'' label
`.'. In these arithmetic expressions everything is implicitly multiplied by the
word size. As such `$.-1$' always refers to the previous operation.

As well as labels, the assembler accepts sections. These sections are used by
the linker to put things where they need to be in the final executable.

% TODO: Describe assembly language in detail.
% TODO: Thorough description of .lab files

\section{Command Usage}

The compiler produces assembly files which when assembled produce LAB object
files. These files contain lists of exported symbols alongside the data they
preface and its length. The linker must then be invoked to put all the symbols
together. The linker contains a default linking file, but custom files can be
used as well.


\subsection{Linker File Description}

A linker file consists of an entry point specification followed by a list of
section descriptions. A section description consists of a file name or glob and
an optional section specification. Alternatively a section description may
specify a modification to the `.' variable, which is the address to which the
next cell of memory will be written.
% TODO: Thorough description of lld files

\subsection{Default Linker File}

The default linker file is as follows:

\begin{lstlisting}
ENTRY (main)
SECTIONS {
    code : { *(.text) };
    initialized_statics : { *(.data) };
    uninitialized_statics : { *(.bss) };
    common : { *(COMMON) };
    anything_else : { * };
    heap_start = .;
}
\end{lstlisting}

\end{document}
