\documentclass{article}

\usepackage[margin=0.5in]{geometry}

\usepackage{amsmath}
\usepackage{array}
\usepackage{booktabs}
\usepackage{caption}
\usepackage{courier}
\usepackage{float}
\usepackage{hyperref}
\usepackage{listings}
\usepackage[usenames]{xcolor}


%{keywords={add,mul,div,mod,and,nan,orr,xor,bgt,beq,cas,bal,ldr,str,get,put,clz,pop,shr,mov,itf,flr,cel,rnd,fad,fmu,fdv,fmo,fbg,fbe,ldf,stf},
\lstdefinelanguage{lab}
{keywords={add,mul,div,mod,adu,mlu,dvu,mdu,and,orr,xor,imp,nan,nor,xnr,lie,jgt,jeq,cmp,jal,jlt,jne,cas,int,ldr,str,get,put,ldb,stb,gtb,ptb,clz,ppz,shr,mov,clo,pop,she,msr,itf,flr,cel,rnd,utf,flu,clu,rdu,fad,fmu,fdv,fmo,exp,pow,log,hyp,sin,cos,tan,cmp,asn,acs,atn,ext},
 otherkeywords={r0,r1,r2,r3,r4,r5,r6,r7,fp,sp,(,)},
 sensitive=false,
 comment=[l]{;},
}

\definecolor{grey}{rgb}{0,0.6,0}
\lstset{
    basicstyle=\ttfamily,
    keywordstyle=\color{blue},
    commentstyle=\color{grey}
}

\newcommand{\labcode}[1]{\colorbox{lightgray}{\lstinline[language=lab]{#1}}}

\title{LAB Processor -- Overview}
\date{\today}
\author{Louis A. Burke}

\begin{document}
\maketitle\clearpage

\section{Introduction}

\section{Registers}

There are 32 registers indexes in the LAB processor. These address 31 registers
and 32 special registers.

The normal registers' values are summarized in \ref{table:registers} while the
special registers' values are summarized in \ref{table:sregisters}. Only
protected mode code can modify the special register values.

\begin{table}[h!]
\centering
\begin{tabular}{ll}
    \toprule Register & Description \\ \midrule
    \labcode{r0} & Read-only, returns 0 \\
    \labcode{r1} .. \labcode{r28} & General purpose \\
    \labcode{fp} (\labcode{r29}) & Frame pointer (works as general purpose) \\
    \labcode{sp} (\labcode{r30}) & Stack pointer (initialized to highest address) \\
    \labcode{pc} (\labcode{r31}) & Program counter \\
\end{tabular}
\caption{LAB Registers}
\label{table:registers}
\end{table}

\begin{table}[h!]
\centering
\begin{tabular}{ll}
    \toprule Register & Description \\ \midrule
    \labcode{sr0} & Interrupt 0 ISR \\
    \labcode{sr1} & Interrupt 1 ISR \\
    \labcode{sr2} & Interrupt 2 ISR \\
    \labcode{sr3} & Interrupt 3 ISR \\
    \labcode{sr4} & Interrupt 4 ISR \\
    \labcode{sr5} & Interrupt 5 ISR \\
    \labcode{sr6} & Interrupt 6 ISR \\
    \labcode{sr7} & Interrupt 7 ISR \\
    \labcode{sr8} & Interrupt 8 ISR \\
    \labcode{sr9} & Interrupt 9 ISR \\
    \labcode{sr10} & Interrupt A ISR \\
    \labcode{sr11} & Interrupt B ISR \\
    \labcode{sr12} & Interrupt C ISR \\
    \labcode{sr13} & Interrupt D ISR \\
    \labcode{sr14} & Interrupt E ISR \\
    \labcode{sr15} & Interrupt F ISR \\
    \labcode{sr16} & Interrupt 0|1 Status (Source/Enable) \\
    \labcode{sr17} & Interrupt 2|3 Status (Source/Enable) \\
    \labcode{sr18} & Interrupt 4|5 Status (Source/Enable) \\
    \labcode{sr19} & Interrupt 6|7 Status (Source/Enable) \\
    \labcode{sr20} & Interrupt 8|9 Status (Source/Enable) \\
    \labcode{sr21} & Interrupt A|B Status (Source/Enable) \\
    \labcode{sr22} & Interrupt C|D Status (Source/Enable) \\
    \labcode{sr23} & Interrupt E|F Status (Source/Enable) \\
    \labcode{III} (\labcode{sr24}) & Invalid Instruction ISR \\
    \labcode{SII} (\labcode{sr25}) & Software Interrupt ISR \\
    % https://stackoverflow.com/questions/6292620/why-are-the-return-addresses-of-prefetch-abort-and-data-abort-different-in-arm-e
    \labcode{PAI} (\labcode{sr26}) & Prefetch Abort ISR (Couldn't load OP - should return to IR-1) \\
    \labcode{DAI} (\labcode{sr27}) & Data Abort ISR (OP couldn't load args - should return to IR-2) \\
    \labcode{OAI} (\labcode{sr28}) & Operation Abort ISR (e.g. divide by zero) \\
    \labcode{ISP} (\labcode{sr29}) & Interrupt SP \\
    \labcode{MPA} (\labcode{sr30}) & Maximum Protected Address (write to call mode switch) \\
    \labcode{PLN} (\labcode{sr31}) & Length of unprotected memory block \\
\end{tabular}
\caption{LAB Special Registers}
\label{table:sregisters}
\end{table}

\section{Operations}

\subsection{Summary}

Some definitions:

$$\mbox{cmp}(X,Y) = \begin{cases}
    1 & \mbox{if } X > Y \\
    -1 & \mbox{if } X < Y \\
    0 & \mbox{otherwise}
\end{cases}$$

\begin{table}[h!]
\centering
\begin{tabular}{cclll}
    \toprule Group & Op-code & OP & Description & Special Mode \\ \midrule
    Arithmetic  & 00000 & \labcode{ADD}/\labcode{ADU} & $A := A + B + I$ & Signed/Unsigned \\
                & 00001 & \labcode{MUL}/\labcode{MLU} & $A := A(B + I)$ & Signed/Unsigned \\
                & 00010 & \labcode{DIV}/\labcode{DVU} & $A := \frac{A}{B + I}$ & Signed/Unsigned \\
                & 00011 & \labcode{MOD}/\labcode{MDU} & $A := A \mod (B + I)$ & Signed/Unsigned \\ \midrule
    Bitwise     & 00100 & \labcode{AND}/\labcode{NAN} & $A := A \land (B + I)$ & Normal/Negated \\
                & 00101 & \labcode{ORR}/\labcode{NOR} & $A := A \lor (B + I)$ & Normal/Negated \\
                & 00110 & \labcode{XOR}/\labcode{XNR} & $A := A \oplus (B + I)$ & Normal/Negated \\
                & 00111 & \labcode{IMP}/\labcode{LIE} & $A := A \rightarrow (B + I)$ & Normal/Negated \\ \midrule
    Branches    & 01000 & \labcode{JGT}/\labcode{JLT} & if $A >$/$< 0$ goto $B + I$ & Normal/Inverted \\
                & 01001 & \labcode{JEQ}/\labcode{JNE} & if $A =$/$\ne 0$ goto $B + I$ & Normal/Inverted \\
                & 01010 & \labcode{CMP}/\labcode{STC} & $A := \mbox{cmp}/\mbox{stc}^\dagger(A,B+I)$ & Compare/SC \\
                & 01011 & \labcode{JAL}/\labcode{INT} & $A := PC$ goto $B + I$ & Normal/Interrupt \\ \midrule
    Memory      & 01100 & \labcode{LDR}/\labcode{LDB} & $A := [B + I]$ & Little/Big-endian \\
                & 01101 & \labcode{STR}/\labcode{STB} & $[B + I] := A$ & Little/Big-endian \\
                % https://stackoverflow.com/questions/3215878/what-are-in-out-instructions-in-x86-used-for
                & 01110 & \labcode{GET}/\labcode{GTB} & $A := \mbox{read}(B + I)$ & Little/Big-endian \\
                & 01111 & \labcode{PUT}/\labcode{PTB} & $\mbox{write}(B + I) := A$ & Little/Big-endian \\ \midrule
    Bit-Manip   & 10000 & \labcode{CLZ}/\labcode{CLO} & $A := \mbox{clz}(B + I)$ & Zeroes/Ones \\
                & 10001 & \labcode{PPZ}/\labcode{POP} & $A := \mbox{popcnt}(B + I)$ & Zeroes/Ones \\
                & 10010 & \labcode{LBT}/\labcode{SBT} & $A := [B + I]_{byte}$/$[B + I]_{byte} := A_{low}$ & Zeroes/Ones \\ % Sign extension via 1-padding
                & 10011 & \labcode{MOV}/\labcode{MSR} & $A := B + I$/$SPR_A := B + I$ & Normal/Special \\ \midrule
    Float I     & 10100 & \labcode{ITF}/\labcode{UTF} & $A := \mbox{float}(B + I)$ & Signed/Unsigned \\
    (Conversion)& 10101 & \labcode{FLR}/\labcode{FLU} & $A := \lfloor B + I \rfloor$ & Signed/Unsigned \\
                & 10110 & \labcode{CEL}/\labcode{CLU} & $A := \lceil B + I \rceil$ & Signed/Unsigned \\
                & 10111 & \labcode{RND}/\labcode{RDU} & $A := \lfloor B + I + 0.5 \rfloor$ & Signed/Unsigned \\ \midrule
    Float II    & 11000 & \labcode{FAD}/\labcode{EXP} & $A := A + B + I$/$e^{B+I} - 1$ & Add/Exponent \\
    (Arithmetic)& 11001 & \labcode{FMU}/\labcode{POW} & $A := A(B + I)$/$A^{B+I}$ & Multiply/Power \\
                & 11010 & \labcode{FDV}/\labcode{LOG} & $A := \frac{A}{B + I}$/$\ln(B + I)$ & Divide/Log \\
                & 11011 & \labcode{FMO}/\labcode{HYP} & $A := A - (B + I) \lfloor \frac{A}{B + I} \rfloor$/$\sqrt{A^2 + (B + I)^2}$ & Normal/Hypot \\ \midrule
    Float III   & 11100 & \labcode{SIN}/\labcode{ASN} & $A := \sin(B + I)$/$\arcsin(B + I)$ & Normal/Inverted \\
    (Control)   & 11101 & \labcode{COS}/\labcode{ACS} & $A := \cos(B + I)$/$\arccos(B + I)$ & Normal/Inverted \\
                & 11110 & \labcode{TAN}/\labcode{ATN} & $A := \tan(B + I)$/$\arctan(A, B + I)$ & Normal/Inverted \\
                & 11111 & \labcode{CMP}/\labcode{LDL} & $A := \mbox{cmp}/\mbox{ldl}(A,B+I)$ & Normal/LL \\ \midrule
\end{tabular}
\caption{LAB Operations}
\label{table:operations}
\end{table}

$^\dagger$ \labcode{STC} and \labcode{LDL} implement a
load-link/store-conditional pair. Load-link reads the value in memory at $[B +
I]$ into $A$. Then a store-conditional to the same address will only write the
value of $A$ to $[B + I]$ if no updates have occurred at that address since the
last load-link there.

\subsection{Arithmetic}

\subsection{Bitwise}

\subsection{Branches}

\subsection{Memory}

\subsection{Manipulation}

\subsection{Float Conversion}

\subsection{Float Arithmetic}

\subsection{Float Control}

\section{Encoding}

Each operation is encoded across as many bits as the register size. The first
bit indicates whether or not to use the 'special' version of the instruction.
The next five bits encode the instructions, then five bits to select both the A
and B registers, then as many bits as are required to index the instruction plus
one indicate the bit shift (in 2's compliment) and finally the remaining bits
encode the immediate value (also in 2's compliment).

This can be viewed as: \texttt{SNNN\_NNAA AAAB\_BBBB SSS... III...} however this
is largely incorrect, as the 'first' bit is actually the least significant. Thus
more accurately it can be viewed as:
\texttt{\textbf{...III ...SSS BBBB\_BAAA AANN\_NNNS}}.

For example, the layout for a 32-bit, 8-bit-byte machine would be:
\texttt{\textbf{IIII\_IIII IISS\_SSSS BBBB\_BAAA AANN\_NNNS}}. While a 64-bit,
8-bit-byte machine would be:
\texttt{\textbf{IIII\_IIII IIII\_IIII IIII\_IIII IIII\_IIII IIII\_IIII ISSS\_SSSS BBBB\_BAAA AANN\_NNNS}}.

The immediate value is signed using 2's compliment notation.

Unlike most processors, memory is accessed by word instead of byte. Thus the
current instruction would be at $PC-1$ instead of $PC-4$ or $PC-8$.

\section{Modes}

The CPU starts in protected mode. In this mode all memory is writable and all
special registers are accessible. The mode will switch to user mode as soon as
the second-last special register is written to. In user mode any attempt to
access memory outside of the unprotected area, or to access the special
registers will result in an invalid instruction interrupt being raised.

The unprotected memory space is denoted with the last two special registers. The
first \labcode{MPA} (\labcode{sr30}) denotes the highest address which is
protected from below, while the second \labcode{PLN} (\labcode{sr31}) indicates
the length in words of the unprotected area. All memory accesses are translated
by adding \labcode{MPA} to the accessed location. Note that this means that
memory 0 will trigger an invalid instruction interrupt ($0 + $
\labcode{MPA}$=$\labcode{MPA}, \labcode{MPA} is protected).

To switch back to protected mode all you have to do is trigger any interrupt.
All interrupts are handled in protected mode.

If all memory is protected (by setting \labcode{PLN} to 0) then the CPU sleeps.
It will reawaken when an interrupt is to be processed.

\section{Interrupts}

\subsection{Processing}

When an interrupt occurs the CPU switches to protected mode. Execution jumps to
the address in the relevant special register. The return address (old program
counter) is stored at the memory location pointed to by the \labcode{ISP}
special register (\labcode{sr29}). Meanwhile, that address is loaded into
the \labcode{sp} register (\labcode{r30}). The value of the \labcode{ISP}
special register is set to the old value of the \labcode{sp} register. When any
value is written to the \labcode{ISP} special register, execution jumps to that
address. This causes the \labcode{sp} register and the \labcode{ISP} special
register to swap their values back.

If the return address is the same as the address before the interrupt, then the
mode is restored, otherwise the mode is switched to protected mode and will only
return to user mode when \labcode{MPA} is written to.

From the time when an interrupt occurs to the time the \labcode{ISP} special
register is written no other interrupts can occur.

\subsection{Configuration}

% TODO: More x86, less ARM (http://wiki.osdev.org/Interrupts)
% Special registers for address, have to manually save required registers

\subsection{Context Switch}

% TODO: Via SWI, return to kernel causes mode switch, other interrupts can occur
% once returned
% INT LR

\subsection{Tools}

This project comes with a number of tools for working with the LAB architecture.
Some are to assist in implementation of LAB machines while some assist in
development for such machines.

\subsubsection{Implementations}
\subsubsection{Linker}
\subsubsection{Assembler}
\subsubsection{C Compiler}

\end{document}
